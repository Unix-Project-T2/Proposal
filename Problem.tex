\section*{Problem}
\label{sec:Problem}

% Begin writing about problem here...

